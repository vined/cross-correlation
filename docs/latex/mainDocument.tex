% Kompiuterijos katedros šablonas
% Template of Department of Computer Science II
% Versija 1.0 2015 m. kovas [ March, 2015]

\documentclass[a4paper,12pt,fleqn]{article}
\input{allPacks}

\newtoggle{inLithuanian}
%If the report is in Lithuanian, it is set to true; otherwise, change to false
\settoggle{inLithuanian}{true}

%create file preface.tex for the preface text
%if preface is needed set to true
\newtoggle{needPreface}
\settoggle{needPreface}{false}

\newtoggle{signaturesOnTitlePage}
\settoggle{signaturesOnTitlePage}{false}


\input{macros}

\begin{document}
    % #1 -report type, #2 - title, #3-7 students, #8 - supervisor
    \depttitlepage{Pirma užduotis}{Signalų poros tarpusavio koreliacija}{Edvinas Naraveckas}
    {}{}{}{}% students 2-5
    {}

    \tableofcontents

    %Introduction section: label is sec:intro
    \sectionWithoutNumber{\keyWordIntroduction}{intro}
    Darbe nagrinėjamos šešios signalų poros ir jų tarpusavio koreliacija.
Užduotis atlikta \textit{Haskell} kalba. Grafikai gauti naudojant \textit{Gnuplot} įrankį.

Jei turme diskrečius signalus \(f(t)\) ir \(g(t)\) kurių laiko trukmė \(t\) yra lygi
ir abiejų signalų diskretizavimo žingsnis \(\Delta t\) yra tas pats,
tai signalų tarpusavio koreliacija apibrėžiama funkcija:

\begin{equation}
    r_{fg} (d) = \dfrac{\displaystyle \sum\limits_{j=0}^{N-d} {(f_j - \bar{f}) (g_{d+j} - \bar{g})}} {\sqrt{\displaystyle \sum\limits_{j=0}^{N-d} (f_j - \bar{f})^2 \displaystyle \sum\limits_{j=0}^{N-d} (g_{d+j} - \bar{g})^2}},\qquad
    d = 0,1, \ldots\ ,[N/2].
\end{equation}

čia

\begin{equation}
    \bar{f} = \dfrac{1} {N-d+1} \displaystyle \sum\limits_{i=0}^{N-d} {f_i},\qquad
    \bar{g} = \dfrac{1} {N-d+1} \displaystyle \sum\limits_{i=0}^{N-d} {g_{d+i}}.
\end{equation}

Parametras \(d\) yra poslinkis laike.
Tai yra, per kiek diskretaus laiko žingsnių yra pastumtos \(g\) reikšmės \(f\) reikšmių atžvilgiu.

Jei sukeičiame \(f\) ir \(g\) reikšmes, galime skaičiuoti neigiamą signalų koreliacią, arba į praeitį.
kas iš tiesų parodo ar \(g\) kitimas turėjo įtakos \(f\) reikšmėms.
Funkciją \(r_{gf}\) galima interpretuoti kaip \(r_{fg}\) funkcijos pratesimą į neigiamas \(d\) reikšmes:

\begin{equation}
    r_{fg} (d) = r_{gf} (|d|),\qquad d = 0,-1, \ldots\ ,-[N/2].
\end{equation}

Grafikas pagal \(d\) nėra pakankamai informatyvus, nes reikia žinoti diskretaus laiko žingsį.
Geresnis būdas atvaizduoti signalų tarpusavio koreliaciją yra naudojant priklausomybę nuo realaus laiko \(t\).
Tuomet signalų koreliacijos funkcija nuo \(t\) žymėsime \(R_{fg}(t)\):

\begin{equation}
    R_{fg} (t) = r_{fg} (d),\qquad t = d\Delta t,\qquad d = -[N/2],-[N/2]+1, \ldots\ ,0,1, \ldots\ ,[N/2].
\end{equation}

Signalų tarpusavio koreliacijos grafikas \(R_{fg}(t)\) rodo signalų panašumą.
Jo rezultatus galima interpretuoti suskirsčius gautas reikšmes intervalais:

\begin{itemize}
    \item Visiems \(t, |R_{fg}(t) \in [0, 0.2]\): tarp signalų \textit{nėra} statistinio panašumo arba jis labai silpnas.
    \item Tam tikriems \(t, |R_{fg}(t) \in (0.2, 0.5]\): tarp signalų yra \textit{silpnas} statistinis ryšys.
    \item Tam tikriems \(t, |R_{fg}(t) \in (0.5, 0.7]\): tarp signalų yra \textit{vidutinis} statistinis ryšys.
    \item Tam tikriems \(t, |R_{fg}(t) \in (0.5, 1]\): tarp signalų yra \textit{stiprus} statistinis ryšys.
\end{itemize}

O, maksimali \(t\) reikšmė parodo per kiek reikėtų pastumti \(g\) reikšmes, kad jie taptų labiausiai panašūs.



    %the main part
    \newpage
    \section{Karo pabėgėliai ir „war“ žodžio dažnumas antraštėse}
    \label{sec:refugees}
    Karo pabėgėliu mėnesinis skaičius pagal šalis\cite{refugees} buvo sutrauktas į bendrą per mėnesinį pasaulio babėgėlių skaičių 2003 - 2017 metais.
Žodžio „war“ dažnumas buvo surinktas iš ABC naujienų portalo antraščių\cite{abcNews} tuo pačiu laikotarpiu.

\includegraphics[scale=0.65]{../scripts/refugees_war/refugees.png}
\includegraphics[scale=0.65]{../scripts/refugees_war/war.png}
\includegraphics[scale=0.65]{../scripts/refugees_war/result.png}


    \section{Saulės dėmių skaičius ir vidutinės temperatūros anomalijos}
    \label{sec:sunspots}
    Vidutinio mėnesinio saulės dėmių skaičiaus duomenys\cite{sunspots} 1880 - 2016 metais lyginami su
globalios mėnesinės temperatūros vidutinės anomalijomis\cite{temp} tuo pačiu laikotarpiu.

\begin{figure}
\includegraphics[scale=0.65]{../scripts/sunspots_temperature/sunspots.png}
\includegraphics[scale=0.65]{../scripts/sunspots_temperature/temp.png}
\includegraphics[scale=0.65]{../scripts/sunspots_temperature/result.png}
\caption{Grafikas kairėje: vidutinis mėnesinis saulės dėmių skaičius. Grafikas centre: globalios mėnesinės temperatūros vidutinės anomalijos. Grafikas dešinėje: signalų tarpusavio koreliacija.}
\end{figure}

Signalų poros koreliacijos funkcija rodo didžiausią signalų panašumą \( R_{fg}(t) = 0.39 \), kai \( t = 692 \).
\( R_{fg}(t)\) rodo silpną statistinį ryšį. Bet \(t\) yra apie 57 metai.
Tai per didelis poslinkis.
Saulės aktyvumas kinta periodiškai kas vienuoliką metų.
Poslinkis \(t\) turėtų būti \(< 122 \) mėnesių.
Esant tokiems rezultatams galima konstatuoti, kad koreliacijos tarp saulės aktyvumo ir temperatūros anomalijų nėra arba reikia smulkesnės analizės.
Panašias išvadas pateikia kitas tyrimas\cite{temp_study}.



    %Conclusions section
    \sectionWithoutNumber{\keyWordConclusions}{conclusion}
    Signalų tarpusavio koreliacijos metodas leidžia nustatyti ar tarp signalų poros yra statistinis ryšis.
O poslinkis leidžia įvertinti kaip greitai pokyčiai viename signale įtakoją kitą signalą.
Iš atliktų metodo taikymų matome, kad signalai gali būti įvairūs ir reikalauja tik minimalių apdorojimo darbų prieš naudojimą.
Signalų dimensijos gali nesutapti ar net turėti visisškai skirtingą kilmę,
kaip \ref{sec:sunspots} skyriuje tirti saulės dėmių skaičiaus ir temperatūros anomalijų duomenys.



    %file literatureSources.bib
    \referenceSources{literatureSources}



    %% this part is optional
    \newpage
    \begin{appendices}
        \section{Pirmojo priedo pavadinimas}
        \label{app:a}
        Pirmojo priedo tekstas ...

    \end{appendices}


\end{document}
