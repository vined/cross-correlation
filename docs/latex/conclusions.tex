Signalų tarpusavio koreliacijos metodas leidžia nustatyti ar tarp signalų poros yra statistinis ryšis.
O poslinkis leidžia įvertinti kaip greitai pokyčiai viename signale įtakoją kitą signalą.
Iš atliktų metodo taikymų matome, kad signalai gali būti įvairūs ir reikalauja tik minimalių apdorojimo darbų prieš naudojimą.
Signalų dimensijos gali nesutapti ar net turėti visisškai skirtingą kilmę,
kaip \ref{sec:sunspots} skyriuje tirti saulės dėmių skaičiaus ir temperatūros anomalijų duomenys.
