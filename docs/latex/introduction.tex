Darbe nagrinėjamos šešios signalų poros ir jų tarpusavio koreliacija.
Užduotis atlikta \textit{Haskell} kalba. Grafikai gauti naudojant \textit{Gnuplot} įrankį.

Jei turme diskrečius signalus \(f(t)\) ir \(g(t)\) kurių laiko trukmė \(t\) yra lygi
ir abiejų signalų diskretizavimo žingsnis \(\Delta t\) yra tas pats,
tai signalų tarpusavio koreliacija apibrėžiama funkcija:

\begin{equation}
    r_{fg} (d) = \dfrac{\displaystyle \sum\limits_{j=0}^{N-d} {(f_j - \bar{f}) (g_{d+j} - \bar{g})}} {\sqrt{\displaystyle \sum\limits_{j=0}^{N-d} (f_j - \bar{f})^2 \displaystyle \sum\limits_{j=0}^{N-d} (g_{d+j} - \bar{g})^2}},\qquad
    d = 0,1, \ldots\ ,[N/2].
\end{equation}

čia

\begin{equation}
    \bar{f} = \dfrac{1} {N-d+1} \displaystyle \sum\limits_{i=0}^{N-d} {f_i},\qquad
    \bar{g} = \dfrac{1} {N-d+1} \displaystyle \sum\limits_{i=0}^{N-d} {g_{d+i}}.
\end{equation}

Parametras \(d\) yra poslinkis laike.
Tai yra, per kiek diskretaus laiko žingsnių yra pastumtos \(g\) reikšmės \(f\) reikšmių atžvilgiu.

Jei sukeičiame \(f\) ir \(g\) reikšmes, galime skaičiuoti neigiamą signalų koreliacią, arba į praeitį.
kas iš tiesų parodo ar \(g\) kitimas turėjo įtakos \(f\) reikšmėms.
Funkciją \(r_{gf}\) galima interpretuoti kaip \(r_{fg}\) funkcijos pratesimą į neigiamas \(d\) reikšmes:

\begin{equation}
    r_{fg} (d) = r_{gf} (|d|),\qquad d = 0,-1, \ldots\ ,-[N/2].
\end{equation}

Grafikas pagal \(d\) nėra pakankamai informatyvus, nes reikia žinoti diskretaus laiko žingsį.
Geresnis būdas atvaizduoti signalų tarpusavio koreliaciją yra naudojant priklausomybę nuo realaus laiko \(t\).
Tuomet signalų koreliacijos funkcija nuo \(t\) žymėsime \(R_{fg}(t)\):

\begin{equation}
    R_{fg} (t) = r_{fg} (d),\qquad t = d\Delta t,\qquad d = -[N/2],-[N/2]+1, \ldots\ ,0,1, \ldots\ ,[N/2].
\end{equation}

Signalų tarpusavio koreliacijos grafikas \(R_{fg}(t)\) rodo signalų panašumą.
Jo rezultatus galima interpretuoti suskirsčius gautas reikšmes intervalais:

\begin{itemize}
    \item Visiems \(t, |R_{fg}(t) \in [0, 0.2]\): tarp signalų \textit{nėra} statistinio panašumo arba jis labai silpnas.
    \item Tam tikriems \(t, |R_{fg}(t) \in (0.2, 0.5]\): tarp signalų yra \textit{silpnas} statistinis ryšys.
    \item Tam tikriems \(t, |R_{fg}(t) \in (0.5, 0.7]\): tarp signalų yra \textit{vidutinis} statistinis ryšys.
    \item Tam tikriems \(t, |R_{fg}(t) \in (0.5, 1]\): tarp signalų yra \textit{stiprus} statistinis ryšys.
\end{itemize}

O, maksimali \(t\) reikšmė parodo per kiek reikėtų pastumti \(g\) reikšmes, kad jie taptų labiausiai panašūs.
